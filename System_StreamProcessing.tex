
\noindent\textbf{Distributed Processing:}
%\subsection{Stream Processing}
Our monitoring algorithms are implemented on Spark.
As discussed in Section~\ref{detection}, a continuously changing graph of keywords and their co-occurrence relationship is built from the text stream.
To improve the efficiency of graph update, we created an index in each partition of the Spark RDD to rapidly locate data records and enable fine-grained incremental updates~\cite{ju2015igraph}.
This alleviates the overhead of data copy and shuffling when updated nodes only account for a small portion of the graph and when updated statistics of each node can be computed from a known portion of data, \ie tweets containing certain keywords in our case.
Experiments show that our method has a speedup of 3.7X for incremental statistics update and 5.4X for incremental PageRank.

%(What's the performance improvement, by how much?)

%\textbf{Incremental Computing:} The data in each partition of RDD is divided into two parts, \textit{raw data part} and \textit{incremental data part}. The incremental data %part is structured by a sequence of blocks, each contains the update messages in a certain time interval. When all the computation tasks relying on some block have been %executed, the data in the block is merged into the raw data part.

%\textbf{Work Balance:} The computation cost of the hot spot during event detection is higher, often proportional to the heat rate. Upon recording such costly data in each %partition and evaluate its heat rate every period of time, we build an optimizing model to rebalance to make all the partitions approximately have total heat rate. According %to the result of the model, partitions exchange data and record this change to ensure the rightness of locating with the index.



