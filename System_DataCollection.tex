

%\subsection{Data Collection}
\noindent\textbf{Data Collection.}
We built a distributed crawler to collect data from Weibo, the largest microblog platform in China.
The crawler continuously collects the latest microblogs published by users preferrably with a large number of followers, \ie opinion leaders.
The crawler has a master/slave architecture.
The master node utilizes key/value store to perform task scheduling.
Slave nodes get the assignments from the master node and crawl data from Weibo.
A task would monitor the reposts and comments of an original tweet and retrieve the repost and comment list, 
from which we can construct the forward graph of each tweet message.
The tasks are scheduled according to posts' priority, which is weighed by the number of reposts and comments.
Each task has a life cycle so that each thread of post can be effectively monitored or terminated.
In order to make full use of computer resources and the limited API bandwidth, we deploy the crawler in KVM virtual machines.
The data is crawled using Weibo API in public privilege. 
Due to recent change on the public API, up to now we are able to collect 3 - 10 million microblogs daily.
