Emerging event monitoring is defined as early event detection, multi-aspect sub-event correlation analysis, event refinement, and event temporal evolution tracking.
Multi-aspect sub-event analysis is done at detection time of events and hence introduced in \emph{event detection} part.
\emph{Event refinement} performs spam filtering before and after texts are distilled into events and performs context enrichment for events to facilitate the tracking of them.
\emph{Event tracking} traces the temporal evolution pattern of events at the scale from minutes to month.

%\subsection{Event Detection}

\noindent\textbf{Event Detection.}
\label{detection}
The event detection method consists of three steps: trending keyword detection, event detection and multi-aspect hierarchical sub-event correlation construction.
An event is essentially defined as a set of weighted keywords, enriched by its context information such as detection time, geographical location, participants and so on.
Based on co-occurrence relationship of keywords, we adopt a similar graph anomaly detection framework as in \cite{yu2013anomalous} for early detection of trending keywords.
The incoming text is tokenized and keywords are defined as nodes and their co-occurrence relationship as edges in the graph.
The text stream is modeled as streaming edge weight updates where edge frequency based on weighted decay is updated with each pair of incoming keywords co-occurrence.
Trends over different time granularities could be monitored simultaneously by setting different decay rate.
The graph stream model is fully distributed with a linear scalability since it follows a data-parallel paradigm where the data could be partitioned according to keywords, \ie each computing node would only need the data containing the keyword in its partition.
An efficient context statistics maintenance strategy is developed for efficient update.
Specifically, for each node $i$ three pieces of information are maintained:
(i) the last time stamp $L(i)$ at which an edge was received of node $i$
(ii) the set of nodes in $S(i,t)$
(iii) an array of frequency values $F(i, j, L(i))$ for each node $j\in S(i,t)$.

A scalable overlapping community detection method is applied to detect events over trending keywords.
Keywords of the same event would co-occur and be more densely linked internally than with keywords from a different event.
It would be intuitive to follow modularity-based~\cite{weng2011event} or betweenness centrality based~\cite{sayyadi2013toit}
community detection methods for event detection.
However, such methods would lead to 4 major drawbacks:
(1) non-local: density changes in parts of the graph would affect the overall result for community detection,
(2) mixed contexts: a keyword cannot appear in different events,
(3) co-occurrence by chance: keywords in different contexts would likely to be assigned to the same event due to lack of explicit definition of keyword community,
(4) no hierarchy: hierarchical correlation analysis is absent for such methods.
Events would naturally consist of different aspects which formulate a hierarchical structure.

To distill the events and their correlations, 
we use \emph{k-clique percolation} to find overlapping communities as events~\cite{palla2005uncovering}.
A \emph{k-clique-community} is defined as a union of all k-cliques that can be reached from each other through a series of
adjacent k-cliques, where adjacency means sharing \emph{k-1} nodes.
Such definition of community enjoys several advantages versus existing methods:
(1) local definition of community that would not be affected by change in parts of the graph, 
(2) natural overlap to decode polysemy and diverse contexts, 
(3) reduce co-occurrence by chance through parameter $k$, 
(4) hierarchical view of sub-events is encoded in the maximal cliques of the communities.

A frequent pattern tree mining technique is adopted to retrieve hierarchy of correlated sub-events from weighted keywords of each event.
Valuable insights, such as different aspects of events, causality of events or categorical structure of events, could be revealed.
We are the first to provide a hierarchical view of correlated sub-events for the emerging event detection scenario.
Highly matching tweets related to events are retrieved as representation texts through querying search engine with keywords of sub-event.

We manually inspect 100 detected real world events and its first related tweet in our dataset.
We find that the method can detect 79\% of events within 10 minutes and 47\% within 5 after the first tweet appeared.
The method can outperform state-of-the-art topic and event detection methods based on keyword
co-occurrence \cite{weng2011event, sayyadi2013toit, yan2013biterm} in terms of keyword coherence (NPMI) and summarization quality (ROUGE-1).
The method is implemented on Spark and share a cluster of 8 machines
with other algorithms and system components, where it can handle 16k tweets per second with linear horizontal scalability.


