\section{Introduction}

Emerging event monitoring over microblog platforms has attracted much attention from both research and application domains due to the real-time nature and viral diffusion of information.
In fact they have many times been the first reporter or even the major hosting venue of significant social events, such as the earthquake forecast and presidential election campaign.
A wide variety of events would emerge from such platforms, ranging from political and daily affairs to natural disasters and public security menace.
Not only are the events that attract mass attention important, but also those events with great potentials to go viral while only drew attention from a small fraction of users, \eg a road stabbing with growing witnesses tweeting about it.
Despite the versatility in topics and traffic, different aspects of events and different angles of user descriptions would complement each other.
Further insights, which we refer to as hierarchical sub-event structures, could be revealed, such as multiple aspects of the event, \eg the history and current development of a political event, or causality of events, \eg the capture of a criminal and the crime he had committed, or categorical information, \eg different genres of news from the same agency.
What's more, trends would emerge from different time granularities from minutes to hours, \eg a concert or sports game, or even days to months, \eg a publicly concerned long trial.
The temporal evolution pattern could be recovered to get the big picture of events.
Emerging event monitoring, \ie early detection, hierarchical correlation analysis, and temporal evolution tracking of these events in real-time, provides valuable insights. % and allow timely response to these events.
Such intelligence are desirable and directly related to government agencies, news groups and marketing strategies, etc.

However, traditional topic modeling methods (e.g. Latent Dirichlet Allocation%~\cite{blei2003latent}
) and its derivatives could not be directly applied to noisy short texts, and are not able to perform early detection of emerging events in real-time.
TwitterMonitor~\cite{ mathioudakis2010twittermonitor} provides online detection for general trending events but could not reveal multiple aspects of events nor track the evolution of them.
CLEar~\cite{xie2014clear} provides real-time event detection and tracking but could not provide hierarchical analysis of events.
Neither of them could identify potential viral events at small scale.
Signitrend~\cite{schubert2014signitrend} could detect potential viral events at small scale but is confined to only detection, and could not provide event tracking and correlation analysis.
The method would also tend to generate trends that only contain a single keyword, which is hard to understand for users.
Another important aspect of event monitoring applications is the system optimization.
None of the above provides the horizontal scalability with distributed computations, nor do they investigate system optimizations for their applications.


In this demonstration, we present \ring, a real-time emerging event monitoring system over microblog text streams that integrates our efforts on both research and development of emerging event monitoring.
Specifically, we develop event detection, event evolution tracking and event refinement algorithms to monitor events in real-time.
We optimize the full-text indexing engine and distributed processing engine to improve efficiency.
We also provide a user friendly interface to visualize and facilitate the analysis of emerging events.
The features of \ring system is as follows.

From the event monitoring perspective, \ring provides
(1) early detection of potential trending events even with small traffic of tweets, long before they go viral. Trends at different time granularities, \ie trends over 10 minutes or 1 hour, could be simultaneously revealed at detection time to facilitate better tracking of events.
(2) multi-aspect hierarchical view of correlated sub-events, e.g. different aspects of events such as highlights of a sports game, causality of events such as the trigger and outcome of an investigation, or categorical structure of events such as different genres of detected news reports.
(3) event evolution tracking to monitor the temporal development of an event and trace of its origin.
(4) context information of events and filtering of spam events.
From a system perspective, \ring provides
(1) horizonal scalability to handle full Weibo or Twitter data.
(2) optimized indexing for efficient ranged query over time for event tracking and user interaction.
(3) optimized distributed processing engine handle large volume of data stream.
(4) friendly user interface to visualize events, contexts and evolutions.

\ring is the first system to enjoy such rich set of event monitoring features with dedicated system optimization efforts.
It is also the first system to provide multi-aspect hierarchical view of correlated sub-events under real-time emerging event monitoring scenario.
The capability of \ring system is demonstrated through functional features such as top trending events, event evolution analysis, event query and context extraction e.g. geographical information.

%\noindent
%\textbf{Related Work.} Xie et al. \cite{xie2014clear} provides real-time online observatory
